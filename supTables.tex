% !TEX TS-program = pdflatex
% !TEX encoding = UTF-8 Unicode

% This is a simple template for a LaTeX document using the "article" class.
% See "book", "report", "letter" for other types of document.

\documentclass[11pt]{article} % use larger type; default would be 10pt

\usepackage[utf8]{inputenc} % set input encoding (not needed with XeLaTeX)

%%% Examples of Article customizations
% These packages are optional, depending whether you want the features they provide.
% See the LaTeX Companion or other references for full information.

%%% PAGE DIMENSIONS
\usepackage{geometry} % to change the page dimensions
\geometry{a4paper} % or letterpaper (US) or a5paper or....
% \geometry{margin=2in} % for example, change the margins to 2 inches all round
% \geometry{landscape} % set up the page for landscape
%   read geometry.pdf for detailed page layout information

\usepackage{graphicx} % support the \includegraphics command and options

% \usepackage[parfill]{parskip} % Activate to begin paragraphs with an empty line rather than an indent

%%% PACKAGES
\usepackage{booktabs} % for much better looking tables
\usepackage{array} % for better arrays (eg matrices) in maths
\usepackage{paralist} % very flexible & customisable lists (eg. enumerate/itemize, etc.)
\usepackage{verbatim} % adds environment for commenting out blocks of text & for better verbatim
\usepackage{subfig}
\usepackage{booktabs}

%%% HEADERS & FOOTERS
\usepackage{fancyhdr} % This should be set AFTER setting up the page geometry
\pagestyle{fancy} % options: empty , plain , fancy
\renewcommand{\headrulewidth}{0pt} % customise the layout...
\lhead{}\chead{}\rhead{}
\lfoot{}\cfoot{\thepage}\rfoot{}

%%% SECTION TITLE APPEARANCE
\usepackage{sectsty}
\allsectionsfont{\sffamily\mdseries\upshape} % (See the fntguide.pdf for font help)
% (This matches ConTeXt defaults)

%%% ToC (table of contents) APPEARANCE
\usepackage[nottoc,notlof,notlot]{tocbibind} % Put the bibliography in the ToC
\usepackage[titles,subfigure]{tocloft} % Alter the style of the Table of Contents
\renewcommand{\cftsecfont}{\rmfamily\mdseries\upshape}
\renewcommand{\cftsecpagefont}{\rmfamily\mdseries\upshape} % No bold!

\newcommand{\thickhline}{%
    \noalign {\ifnum 0=`}\fi \hrule height 1.4pt
    \futurelet \reserved@a \@xhline
}

%%% END Article customizations

%%% The "real" document content comes below...

\title{Brief Article}
\author{The Author}
%\date{} % Activate to display a given date or no date (if empty),
         % otherwise the current date is printed 

\begin{document}
\maketitle



			\subsection{Lindell-Pinkas 2010} \label{sub:LP_2010_Results_Analysis}

				\FloatBarrier
				\noindent \textbf{32-bit Addition}
				\begin{figure}[!ht]
					\begin{tabular}{| p{4.3cm} | c c c c |}
						\hline
						\textbf{Builder} & \textbf{CPU Time} & \textbf{Wall Time} & \textbf{Bytes Sent} & \textbf{Bytes Recv} \\
						\thickhline
						Input generation & $0.18$ & $0.02$ & $0$ & $0$ \\
						\hline
						Building circuits & $20.39$ & $2.69$ & $0$ & $0$ \\
						\hline
						OT- Sender & $82.75$ & $11.80$ & $1,214,877$ & $655,111$ \\
						\hline
						Sending circuits/commits & $0.63$ & $2.43$ & $6,125,691$ & $0$ \\
						\hline
						Open check circuits & $3.19$ & $3.20$ & $289,396$ & $2,214$ \\
						\hline
						Prove input consistency & $6.82$ & $7.27$ & $18,110$ & $79,784$ \\
						\thickhline
						Total & $113.96$ & $27.41$ & $7,648,074$ & $737,109$ \\
						\hline
					\end{tabular}
					\caption{The performance of the Builder in the Lindell-Pinkas 2010 protocol evaluating the 32-bit addition circuit averaged over 100 trials. \label{table:LP_2010_Add_Builder}}
				\end{figure}
					
				\begin{figure}[!ht]
					\begin{tabular}{| p{4.3cm} | c c c c |}
						\hline
						\textbf{Executor} & \textbf{CPU Time} & \textbf{Wall Time} & \textbf{Bytes Sent} & \textbf{Bytes Recv} \\
						\thickhline
						OT prep receiver & $17.60$ & $2.60$ & $0$ & $0$ \\
						\hline
						OT transfer receiver & $18.93$ & $14.18$ & $655,111$ & $1,214,877$ \\
						\hline
						Receive circuits/commits & $0.40$ & $0.05$ & $0$ & $6,125,691$ \\
						\hline
						Checking correctness & $11.57$ & $3.20$ & $2,214$ & $289,396$ \\
						\hline
						Verify input consistency & $6.87$ & $7.28$ & $79,784$ & $18,110$ \\
						\hline
						Evaluate circuits & $0.03$ & $0.03$ & $0$ & $0$ \\
						\thickhline
						Total & $55.90$ & $27.45$ & $737,109$ & $7,648,074$ \\
						\hline
					\end{tabular}
					\caption{The performance of the Executor in the Lindell-Pinkas 2010 protocol evaluating the 32-bit addition circuit averaged over 100 trials.\label{table:LP_2010_Add_Executor}}
				\end{figure}
				\FloatBarrier

				As expected the Oblivious Transfers dominates the running time of the protocol on such a small circuit with relatively many input wires, especially for the Executing party as it does not bear the burden of building all the circuits. The communication costs are dominated by sending circuits. In future our implementation could be significantly improved by using the circuit hash trick for circuit correctness.\\


				\FloatBarrier
				\noindent \textbf{32-bit Multiplication}
				\begin{figure}[!ht]
					\begin{tabular}{| p{4.3cm} | c c c c |}
						\hline
						\textbf{Builder} & \textbf{CPU Time} & \textbf{Wall Time} & \textbf{Bytes Sent} & \textbf{Bytes Recv} \\
						\thickhline
						Input generation & $0.18$ & $0.02$ & $0$ & $0$ \\
						\hline
						Building circuits & $31.53$ & $4.21$ & $0$ & $0$ \\
						\hline
						OT- Sender & $82.70$ & $11.82$ & $1,214,877$ & $655,111$ \\
						\hline
						Sending circuits/commits & $1.01$ & $4.10$ & $202,012,291$ & $0$ \\
						\hline
						Open check circuits & $3.19$ & $3.22$ & $289,396$ & $2,214$ \\
						\hline
						Prove input consistency & $6.82$ & $7.27$ & $18,109$ & $79,784$ \\
						\thickhline
						Total & $125.43$ & $30.65$ & $203,534,673$ & $737,109$ \\
						\hline
					\end{tabular}

					\caption{The performance of the Builder in the Lindell-Pinkas 2010 protocol evaluating the 32-bit multiplication averaged over 100 trials. \label{table:LP_2010_Mul_Builder}}
				\end{figure}
				
				\begin{figure}[!ht]
					\begin{tabular}{| p{4.3cm} | c c c c |}
						\hline
						\textbf{Executor} & \textbf{CPU Time} & \textbf{Wall Time} & \textbf{Bytes Sent} & \textbf{Bytes Recv} \\
						\thickhline
						OT prep receiver & $17.59$ & $2.60$ & $0$ & $0$ \\
						\hline
						OT transfer receiver & $18.69$ & $14.20$ & $655,111$ & $1,214,877$ \\
						\hline
						Receive circuits/commits & $1.65$ & $1.75$ & $0$ & $202,012,291$ \\
						\hline
						Checking correctness & $17.14$ & $3.19$ & $2,214$ & $289,396$ \\
						\hline
						Verify input consistency & $6.86$ & $7.28$ & $79,784$ & $18,109$ \\
						\hline
						Evaluate circuits & $0.84$ & $0.84$ & $0$ & $0$ \\
						\thickhline
						Total & $63.53$ & $31.50$ & $737,109$ & $203,534,673$ \\
						\hline
					\end{tabular}
					\caption{The performance of the Executor in the Lindell-Pinkas 2010 protocol evaluating the 32-bit multiplication averaged over 100 trials. \label{table:LP_2010_Mul_Executor}}
				\end{figure}
				\FloatBarrier

				The Multiplication circuit takes only a little longer to run, and our prediction with regards to the Oblivious Transfer remaining steady are borne out. Most of the extra time is spent Building the significantly bigger circuits, but this increase is not linear with the increase in the circuit size. We shall see if this pattern continues with the AES circuit.\\

				The main difference overall is in the bandwidth usage. We send about $25$ times more than we did for the Addition circuit.\\

				\pagebreak
				\FloatBarrier
				\noindent \textbf{AES Encryption}
				\begin{figure}[!ht]
					\begin{tabular}{| p{4.3cm} | c c c c |}
						\hline
						\textbf{Builder} & \textbf{CPU Time} & \textbf{Wall Time} & \textbf{Bytes Sent} & \textbf{Bytes Recv} \\
						\thickhline
						Input generation & $0.36$ & $0.05$ & $0$ & $0$ \\
						\hline
						Building circuits & $114.32$ & $15.39$ & $0$ & $0$ \\
						\hline
						OT- Sender & $323.83$ & $42.42$ & $4,859,037$ & $2,477,191$ \\
						\hline
						Sending circuits/commits & $1.37$ & $14.81$ & $663,253,047$ & $0$ \\
						\hline
						Open check circuits & $13.12$ & $13.17$ & $751,156$ & $2,214$ \\
						\hline
						Prove input consistency & $27.82$ & $29.15$ & $72,444$ & $319,112$ \\
						\thickhline
						Total & $480.82$ & $114.98$ & $668,935,684$ & $2,798,517$ \\
						\hline
					\end{tabular}
					\caption{The performance of the Builder in the Lindell-Pinkas 2010 protocol evaluating the AES encryption circuit averaged over 100 trials. \label{table:LP_2010_AES_Builder}}
				\end{figure}

				\begin{figure}[!ht]
					\begin{tabular}{| p{4.3cm} | c c c c |}
						\hline
						\textbf{Executor} & \textbf{CPU Time} & \textbf{Wall Time} & \textbf{Bytes Sent} & \textbf{Bytes Recv} \\
						\thickhline
						OT prep receiver & $67.53$ & $9.03$ & $0$ & $0$ \\
						\hline
						OT transfer receiver & $70.98$ & $51.55$ & $2,477,191$ & $4,859,037$ \\
						\hline
						Receive circuits/commits & $3.16$ & $5.72$ & $0$ & $663,253,047$ \\
						\hline
						Checking correctness & $56.90$ & $13.15$ & $2,214$ & $751,156$ \\
						\hline
						Verify input consistency & $27.45$ & $29.14$ & $319,112$ & $72,444$ \\
						\hline
						Evaluate circuits & $1.15$ & $1.15$ & $0$ & $0$ \\
						\thickhline
						Total & $227.91$ & $116.15$ & $2,798,517$ & $668,935,684$ \\
						\hline
					\end{tabular}
					\caption{The performance of the Executor in the Lindell-Pinkas 2010 protocol evaluating the AES encryption circuit averaged over 100 trials. \label{table:LP_2010_AES_Executor}}
				\end{figure}
				\FloatBarrier

				In the final circuit test for the Lindell-Pinkas Protocol we see the pattern of a fairly linear increase in the communications continue. Similarly we see a fairly linear increase in the cost of the Oblivious Transfers in line with the increase in the number of inputs.\\

			\noindent\textbf{Summary}\\

				Overall most components seem to grow linearly with their input sizes. It is worth noting that the computational costs are heavily imbalanced with the Builder  using around twice as much CPU time as the Executor in all three experiments.\\

				Most of this additional CPU time seems to be spent on the Oblivious Transfers. Communication costs are dominated by the circuits, and so is skewed heavily towards the Builder Sending. This could be greatly reduced using the Hashed circuits trick.\\

				Whilst further experimentation would be needed to check this, it appears the cost of the protocol is primarily linked to the number of Inputs. Further experiments could also test which input size (Builder's or Executor's) has the greatest impact, I suggest it will be the Builders due to the consistency proof.\\

				Lastly we make a side notes with an eye to improving our implementation. The main obvious area for improvement would be the consistency proving stage. At the moment it takes one input wire at a time and therefore does not make use of any extra cores available. This should not be overly taxing however time constraints have prevented us from attempting it. Preliminary investigation (comparing to the ZKPoK in the modified CnC OT) suggest this will speed up the consistency proof by a factor of 4.

			\pagebreak
			\subsection{Lindell 2013} \label{sub:L-2013_Results_Analysis}
				We do not provide a breakdown of the sub-computation here, this comes later when comparing the sub-computation performance for Lindell 2013 and our variant.\\

				\FloatBarrier
				\noindent \textbf{32-bit Addition}
				\begin{figure}[!ht]
					\begin{tabular}{| p{4.3cm} | c c c c |}
						\hline
						\textbf{Builder} & \textbf{CPU Time} & \textbf{Wall Time} & \textbf{Bytes Sent} & \textbf{Bytes Recv} \\
						\thickhline
						Input generation & $0.09$ & $0.01$ & $0$ & $0$ \\
						\hline
						Build circuits/commits & $6.49$ & $0.81$ & $0$ & $0$ \\
						\hline
						OT- Sender & $37.83$ & $8.98$ & $284,040$ & $135,781$ \\
						\hline
						Send circuits/commits & $0.00$ & $0.02$ & $1,889,281$ & $0$ \\
						\hline
						Partially open J-set & $0.99$ & $0.99$ & $88,982$ & $692$ \\
						\hline
						Sub-computation & $117.53$ & $21.85$ & $2,412,286$ & $746,955$ \\
						\hline
						Send B-Lists & $0.00$ & $0.00$ & $1,060$ & $0$ \\
						\hline
						Prove input consistency & $8.27$ & $9.37$ & $18,112$ & $96,764$ \\
						\thickhline
						Total & $171.21$ & $42.03$ & $4,693,761$ & $980,193$ \\
						\hline
					\end{tabular}
					\caption{The performance of the Builder in the Lindell 2013 protocol evaluating the 32-bit addition averaged over 100 trials. \label{table:L_2013_Add_Builder}}
				\end{figure}

				\begin{figure}[!ht]
					\begin{tabular}{| p{4.3cm} | c c c c |}
						\hline
						\textbf{Executor} & \textbf{CPU Time} & \textbf{Wall Time} & \textbf{Bytes Sent} & \textbf{Bytes Recv} \\
						\thickhline
						OT - Receiver & $35.46$ & $9.79$ & $135,781$ & $284,040$ \\
						\hline
						Receive circuits/commits & $0.01$ & $0.03$ & $0$ & $1,889,281$ \\
						\hline
						Partially open J-set & $0.03$ & $0.99$ & $692$ & $88,982$ \\
						\hline
						Evaluate circuits & $0.01$ & $0.01$ & $0$ & $0$ \\
						\hline
						Sub-computation & $53.37$ & $21.85$ & $746,955$ & $2,412,286$ \\
						\hline
						Verify B-List & $0.00$ & $0.00$ & $0$ & $1,060$ \\
						\hline
						Checking correctness & $3.63$ & $0.58$ & $0$ & $0$ \\
						\hline
						Verify input consistency & $9.16$ & $8.79$ & $96,764$ & $18,112$ \\
						\thickhline
						Total & $101.69$ & $42.05$ & $980,193$ & $4,693,761$ \\
						\hline
					\end{tabular}
					\caption{The performance of the Executor in the Lindell 2013 protocol evaluating the 32-bit addition averaged over 100 trials. \label{table:L_2013_Add_Executor}}
				\end{figure}
				\FloatBarrier

				As predicted the Sub-computation dominates the running time. We also see that for such a small circuit the sub-computation is the biggest single factor in the communication costs as well, this will not hold as the circuit size increases. The verification of the B-Lists against the Hashed B-list has a negligible cost and in future this measurement could be folded into the correctness check.\\

				We should now note that we expect the running time of the sub-computation to be about the same for the multiplication circuit as here, and only slightly increased for the AES circuit.\\

				\FloatBarrier
				\noindent \textbf{32-bit Multiplication}
				\begin{figure}[!ht]
					\begin{tabular}{| p{4.3cm} | c c c c |}
						\hline
						\textbf{Builder} & \textbf{CPU Time} & \textbf{Wall Time} & \textbf{Bytes Sent} & \textbf{Bytes Recv} \\
						\thickhline
						Input generation & $0.09$ & $0.01$ & $0$ & $0$ \\
						\hline
						Build circuits/commits & $9.76$ & $1.24$ & $0$ & $0$ \\
						\hline
						OT- Sender & $37.80$ & $8.99$ & $284,040$ & $135,780$ \\
						\hline
						Sending circuits/commits & $0.12$ & $0.52$ & $62,163,073$ & $0$ \\
						\hline
						Partially Open J-sets & $1.01$ & $1.04$ & $89,081$ & $681$ \\
						\hline
						Sub-computation & $117.91$ & $22.38$ & $2,412,286$ & $746,955$ \\
						\hline
						Send B-Lists & $0.00$ & $0.00$ & $2,052$ & $0$ \\
						\hline
						Prove Input Consistency & $8.31$ & $9.67$ & $18,111$ & $97,167$ \\
						\thickhline
						Total & $175.00$ & $43.86$ & $64,968,644$ & $980,583$ \\
						\hline
					\end{tabular}
					\caption{The performance of the Builder in the Lindell 2013 protocol evaluating the 32-bit multiplication circuit averaged over 100 trials. \label{table:L_2013_Mul_Builder} }
				\end{figure}
				
				\begin{figure}[!ht]
					\begin{tabular}{| p{4.3cm} | c c c c |}
						\hline
						\textbf{Executor} & \textbf{CPU Time} & \textbf{Wall Time} & \textbf{Bytes Sent} & \textbf{Bytes Recv} \\
						\thickhline
						OT - Receiver & $35.46$ & $10.23$ & $135,780$ & $284,040$ \\
						\hline
						Receiving circuits/commitments & $0.28$ & $0.56$ & $0$ & $62,163,073$ \\
						\hline
						Partially open J-set & $0.03$ & $1.01$ & $681$ & $89,081$ \\
						\hline
						Evaluate circuits & $0.26$ & $0.26$ & $0$ & $0$ \\
						\hline
						Sub-computation & $53.44$ & $22.13$ & $746,955$ & $2,412,286$ \\
						\hline
						Verify B-List & $0.00$ & $0.00$ & $0$ & $2,052$ \\
						\hline
						Checking correctness & $5.58$ & $0.85$ & $0$ & $0$ \\
						\hline
						Verify input consistency & $9.19$ & $8.82$ & $97,167$ & $18,111$ \\
						\thickhline
						Total & $104.28$ & $43.88$ & $980,583$ & $64,968,644$ \\
						\hline
					\end{tabular}
					\caption{The performance of the Executor in the Lindell 2013 protocol evaluating the 32-bit multiplication circuit averaged over 100 trials.\label{table:L_2013_Mul_Executor} }
				\end{figure}
				\FloatBarrier

				Once again we see the cost of the Oblivious Transfers remaining steady when moving from the addition circuit to the multiplication circuit and a slight increase in the time taken to build the circuits. \\

				The costs of the sub-computation have not significantly changed when compared to the addition circuit, this was  as none of its input sizes have changed. In fact the overall running time has barely changed, this suggests that this protocol will be of most use when evaluating large circuit that have few inputs.\\

				This is hardly a big revelation though, this is more of a trend with Yao based protocols in general that we see with other protocols too.\\


				\FloatBarrier
				\noindent \textbf{AES Encryption}
				\begin{figure}[!ht]
					\begin{tabular}{| p{4.3cm} | c c c c |}
						\hline
						\textbf{Builder} & \textbf{CPU Time} & \textbf{Wall Time} & \textbf{Bytes Sent} & \textbf{Bytes Recv} \\
						\thickhline
						Input generation & $0.27$ & $0.03$ & $0$ & $0$ \\
						\hline
						Build circuits/commits & $36.24$ & $4.58$ & $0$ & $0$ \\
						\hline
						OT- Sender & $142.05$ & $33.20$ & $1,120,488$ & $476,292$ \\
						\hline
						Sending circuits/ommits & $0.33$ & $1.75$ & $204,095,057$ & $0$ \\
						\hline
						Partially open J-set & $4.01$ & $4.05$ & $227,146$ & $701$ \\
						\hline
						Sub-computation & $183.19$ & $37.52$ & $5,018,302$ & $746,955$ \\
						\hline
						Send B-Lists & $0.00$ & $0.00$ & $4,100$ & $0$ \\
						\hline
						Prove input consistency & $33.18$ & $38.11$ & $72,444$ & $385,743$ \\
						\thickhline
						Total & $399.27$ & $119.25$ & $210,537,538$ & $1,609,692$ \\
						\hline
					\end{tabular}
					\caption{The performance of the Builder in the Lindell 2013 protocol evaluating the AES encryption circuit averaged over 100 trials. \label{table:L_2013_AES_Builder}}
				\end{figure}

				\begin{figure}[!ht]
					\begin{tabular}{| p{4.3cm} | c c c c |}
						\hline
						\textbf{Executor} & \textbf{CPU Time} & \textbf{Wall Time} & \textbf{Bytes Sent} & \textbf{Bytes Recv} \\
						\thickhline
						OT - Receiver & $138.36$ & $37.80$ & $476,292$ & $1,120,488$ \\
						\hline
						Receive circuits/commits & $0.77$ & $1.80$ & $0$ & $204,095,057$ \\
						\hline
						Partially open J-Set & $0.04$ & $4.02$ & $701$ & $227,146$ \\
						\hline
						Evaluate circuits & $0.35$ & $0.35$ & $0$ & $0$ \\
						\hline
						Sub-computation & $78.95$ & $37.20$ & $746,955$ & $5,018,302$ \\
						\hline
						Verify B-List & $0.00$ & $0.00$ & $0$ & $4,100$ \\
						\hline
						Checking correctness & $18.36$ & $3.13$ & $0$ & $0$ \\
						\hline
						Verify input consistency & $34.05$ & $34.96$ & $385,743$ & $72,444$ \\
						\thickhline
						Total & $270.99$ & $119.27$ & $1,609,692$ & $210,537,538$ \\
						\hline
					\end{tabular}
					\caption{The performance of the Executor in the Lindell 2013 protocol evaluating the AES encryption circuit averaged over 100 trials. \label{table:L_2013_AES_Executor}}
				\end{figure}
				\FloatBarrier

				We now get to see how the sub-computation fares in a larger circuit. The cost of the sub-computation increases but by a much smaller factor than the rest of the computation, this is hopeful for applying this protocol to larger circuits.\\

				\noindent\textbf{Summary}\\

					It appears that Lindell's suggestion that this protocol is ill-suited to small circuits is entirely correct. As the circuits get bigger the relative cost of the sub-computation falls and this protocol becomes more useful.\\

					The sub-computation is very expensive (even on the AES circuit around $\frac{1}{3}$ of running time is spent in the sub-computation) so this is an area of the protocol ripe for optimisation.\\

					The obvious avenue for further optimisation will be exchanging the protocol used for the sub-computation, something we have attempted by using the HKE protocol for the sub-computation. This said this method can only take use so far, after all any protocol used for the sub-computation could simply be used for the main computation instead. A method of recovering the Builder's input without having to run a full Yao-Circuit computation would be preferable. \\

					As is usual in asymmetric Yao based protocols the computational load is heavier on the Builder and the Builder sends more data than the Executor. When originally thinking about circuits with differing inputs sizes for the two parties we expected we would be aiming to minimise the Executor's inputs to reduce the number of Oblivious Transfers.\\

					The opposite is true here, due to the expensive proofs for builder input consistency we should aim to minimise the number of inputs for the Builder.


			\subsection{Huang-Katz-Evans 2013} \label{sub:HKE_Results_Analysis}

				\FloatBarrier
				\noindent \textbf{32-bit Addition}
				\begin{figure}[!ht]
					\begin{tabular}{| p{4.3cm} | c c c c |}
						\hline
						 & \textbf{CPU Time} & \textbf{Wall Time} & \textbf{Bytes Sent} & \textbf{Bytes Recv} \\
						\thickhline
						Circuits prep. & $2.81$ & $0.37$ & $77$ & $77$ \\
						\hline
						Building circuits & $0.14$ & $0.02$ & $0$ & $0$ \\
						\hline
						Exchanging Circuits & $0.31$ & $0.04$ & $2,162,460$ & $2,162,460$ \\
						\hline
						Exchange VSS schemes & $0.03$ & $0.00$ & $37,422$ & $37,405$ \\
						\hline
						Naor Pinkas OT & $12.37$ & $1.55$ & $156,904$ & $156,904$ \\
						\hline
						Make/Send commits & $12.59$ & $1.64$ & $430,126$ & $430,126$ \\
						\hline
						Coin flip for J-set & $0.08$ & $0.01$ & $2,532$ & $2,532$ \\
						\hline
						Initial J-set checks & $14.90$ & $2.41$ & $313,732$ & $313,732$ \\
						\hline
						Logarithm Checks & $1.08$ & $0.23$ & $26,500$ & $26,500$ \\
						\hline
						Output Determination & $1.16$ & $0.50$ & $13,630$ & $13,630$ \\
						\thickhline
						Total & $45.59$ & $6.77$ & $3,143,383$ & $3,143,366$ \\
						\hline
					\end{tabular}
					\caption{The performance of the Huang-Katz-Evans 2013 protocol evaluating the 32-bit addition circuit averaged over 100 trials.\label{table:HKE_2013_Add}}
				\end{figure}

				The results for the Huang-Katz-Evans protocol are very promising. We have a very high CPU time to Wall time ratio (close to 7 with 8 cores available), indicating good parallelism. In particular the Naor-Pinkas OT step and the Commitment manufacturing step both seem to make good use of all available cores, though to be sure we'd need to run with differing numbers of cores.\\ 

				We also see the computational load is well balanced between the parties (as is communication traffic). In situations where the parties are equally capable this is a definite advantage, in cases where one party is more capable it could be a hindrance.\\

				There is no one step of the protocol that clearly dominates running time, though the initial J-set checks take the longest. Whilst I do not have direct data to back up this hypothesis I believe the opening of commitments is the primary cost in the J-set checks. In terms of communications the sending of circuits is the largest single factor as usual.\\

				\FloatBarrier
				\noindent \textbf{32-bit Multiplication}
				\begin{figure}[!ht]
					\begin{tabular}{| p{4.3cm} | c c c c |}
						\hline
						 & \textbf{CPU Time} & \textbf{Wall Time} & \textbf{Bytes Sent} & \textbf{Bytes Recv} \\
						\thickhline
						Circuits prep. & $2.83$ & $0.38$ & $77$ & $77$ \\
						\hline
						Building circuits & $4.16$ & $0.55$ & $0$ & $0$ \\
						\hline
						Exchanging Circuits & $1.20$ & $1.25$ & $71,476,180$ & $71,476,180$ \\
						\hline
						Exchange VSS schemes & $0.00$ & $0.03$ & $72,529$ & $72,507$ \\
						\hline
						Naor Pinkas OT & $12.56$ & $1.62$ & $156,904$ & $156,904$ \\
						\hline
						Make/Send commits & $12.73$ & $1.67$ & $430,126$ & $430,126$ \\
						\hline
						Coin flip for J-set & $0.10$ & $0.01$ & $2,532$ & $2,532$ \\
						\hline
						Initial J-set checks & $17.36$ & $2.79$ & $342,252$ & $342,252$ \\
						\hline
						Logarithm Checks & $1.08$ & $0.22$ & $26,500$ & $26,500$ \\
						\hline
						Output Determination & $0.85$ & $0.87$ & $26,154$ & $26,154$ \\
						\thickhline
						Total & $54.00$ & $9.73$ & $72,533,254$ & $72,533,232$ \\
						\hline
					\end{tabular}
					\caption{The performance of the Huang-Katz-Evans 2013 protocol evaluating the 32-bit multiplication circuit averaged over 100 trials.\label{table:HKE_2013_Mul}}
				\end{figure}

				The multiplication circuit results are particularly interesting for Huang-Katz-Evans. We predicted that we would see a significant increase in the cost of the output determination due the increase in the number of outputs. We do see a increase in running time though it is slightly below that which we expected, more interestingly there is a \emph{decrease} in CPU time used.\\

				We see the expected increase in cost for the circuit building and the expected lack of change in the cost of the OTs/Commitments/Log checks (all are dependant only on input size). Further we see some support for my hypothesis that the opening of commitments is the main cost in the J-set checks, as the cost goes up only marginally despite the large increase in circuit size (and so circuit correctness check costs).\\

				\FloatBarrier
				\noindent \textbf{AES Encryption}
				\begin{figure}[!ht]
					\begin{tabular}{| p{4.3cm} | c c c c |}
						\hline
						& \textbf{CPU Time} & \textbf{Wall Time} & \textbf{Bytes Sent} & \textbf{Bytes Recv} \\
						\thickhline
						Circuits prep. & $10.91$ & $1.47$ & $77$ & $77$ \\
						\hline
						Building circuits & $12.83$ & $1.72$ & $0$ & $0$ \\
						\hline
						Exchanging Circuits & $2.02$ & $4.11$ & $234,679,488$ & $234,679,488$ \\
						\hline
						Exchange VSS Schemes & $0.00$ & $0.05$ & $144,962$ & $144,967$ \\
						\hline
						Naor Pinkas OT & $49.33$ & $6.37$ & $627,592$ & $627,592$ \\
						\hline
						Make/Send commits & $49.63$ & $6.65$ & $1,719,598$ & $1,719,598$ \\
						\hline
						Coin flip for J-set & $0.07$ & $0.01$ & $2,532$ & $2,532$ \\
						\hline
						Initial J-set checks & $54.58$ & $9.62$ & $968,588$ & $968,588$ \\
						\hline
						Logarithm Checks & $3.17$ & $0.78$ & $105,988$ & $105,988$ \\
						\hline
						Output Determination & $1.70$ & $1.73$ & $52,010$ & $52,010$ \\
						\thickhline
						Total & $185.47$ & $32.95$ & $238,300,835$ & $238,300,840$ \\
						\hline
					\end{tabular}
					\caption{The performance of the Huang-Katz-Evans 2013 protocol evaluating the AES encryption circuit averaged over 100 trials.\label{table:HKE_2013_AES}}
				\end{figure}

				When evaluating the AES circuit we see the expected increase in the time spent on input generation and circuit building. The increase in the cost of the OTs, Commitment construction and Logarithm checks is linear in terms of the growth in the number of inputs. We also see the the continuing trend of the circuit exchange dominating communications costs.\\

				\noindent\textbf{Summary}\\

				The Huang-Katz-Evans protocol shows great promise with good performance which does not seem overly affected by an increase in circuit size nor the number of outputs. Input size affects the running time of the Oblivious Transfer and Commitments, but the impact of increasing the input sizes on performance it seems to be fairly linear.\\

				In a few places our implementation could be improved, mainly the communications code, in several communication rounds we have taken the easy route of $P_1$ going first and having $P_2$ wait till it receives till serialising its own input. This should be changed so the parties serialise symmetrically then take turns to send.\\

				We suggest that further experimentation should be carried out, focusing on parties with unequal computational power but also on circuits with unequal input sizes. Further measurements should also be taken, breaking down the step `Initial J-set Checks'. Whilst we have inferred the most expensive component is likely the opening of commitments this claim needs to be supported by hard data.

			\subsection{L-HKE 2015} \label{sub:L-HKE_Results_Analysis}
				\FloatBarrier
				\noindent \textbf{32-bit Addition}
				\begin{figure}[!ht]
					\begin{tabular}{| p{4.3cm} | c c c c |}
						\hline
						\textbf{Builder} & \textbf{CPU Time} & \textbf{Wall Time} & \textbf{Bytes Sent} & \textbf{Bytes Recv} \\
						\thickhline
						Input generation & $2.34$ & $0.30$ & $77$ & $77$ \\
						\hline
						Circuit Building & $0.12$ & $0.02$ & $0$ & $0$ \\
						\hline
						OT- Sender & $37.65$ & $8.93$ & $284,040$ & $135,784$ \\
						\hline
						Sending circuits, queries and hashes & $0.02$ & $0.03$ & $1,883,800$ & $0$ \\
						\hline
						Make/Send commits & $10.83$ & $1.38$ & $374,062$ & $0$ \\
						\hline
						Partially Open J-Sets & $0.05$ & $0.01$ & $134,368$ & $683$ \\
						\hline
						Sub-computation & $94.77$ & $14.80$ & $3,294,892$ & $3,162,854$ \\
						\hline
						Send B-Lists & $0.00$ & $0.00$ & $1,060$ & $0$ \\
						\hline
						Prove Input Consistency & $0.00$ & $0.00$ & $23,067$ & $0$ \\
						\thickhline
						Total & $145.77$ & $25.47$ & $5,995,366$ & $3,299,399$ \\
						\hline
					\end{tabular}
					\caption{The performance of the Builder in the L-HKE 2015 protocol evaluating the 32-bit addition circuit averaged over 100 trials. \label{table:L-HKE_2015_Add_Builder}}
				\end{figure}

				\begin{figure}[!ht]
					\begin{tabular}{| p{4.3cm} | c c c c |}
						\hline
						\textbf{Executor} & \textbf{CPU Time} & \textbf{Wall Time} & \textbf{Bytes Sent} & \textbf{Bytes Recv} \\
						\thickhline
						OT - Receiver & $35.29$ & $9.23$ & $135,861$ & $284,117$ \\
						\hline
						Receiving circuits and Hashed List & $0.01$ & $0.05$ & $0$ & $1,883,800$ \\
						\hline
						Receiving commits & $0.00$ & $1.39$ & $0$ & $374,062$ \\
						\hline
						Partially open J-set & $0.00$ & $0.00$ & $683$ & $134,368$ \\
						\hline
						Evaluate Circuits & $0.01$ & $0.01$ & $0$ & $0$ \\
						\hline
						Sub-computation & $94.53$ & $14.79$ & $3,162,854$ & $3,294,892$ \\
						\hline
						Verify B-List & $0.00$ & $0.00$ & $0$ & $1,060$ \\
						\hline
						Checking correctness & $1.74$ & $0.24$ & $0$ & $0$ \\
						\hline
						Verify input consistency & $0.93$ & $0.12$ & $0$ & $23,067$ \\
						\thickhline
						Total & $132.51$ & $25.83$ & $3,299,399$ & $5,995,366$ \\
						\hline
					\end{tabular}
					\caption{The performance of the Executor in the L-HKE 2015 protocol evaluating the 32-bit addition circuit averaged over 100 trials. \label{table:L-HKE_2015_Add_Executor}}
				\end{figure}
				\FloatBarrier

				The sub-computation again dominates both running time and communications for a small circuit. The main computation circuits form the next biggest communication cost. We fully expect that as the circuit and input sizes increase the circuit sending we become the largest communication cost.\\

				The logarithm calculation stage for the builder has such a small cost in terms of CPU/Wall time our timing code does not register the time spent. This is not all that odd as for the Builder this is just computing a few subtractions.\\

				It is notable that the major computational costs here are dependant on the size of one the party's input. The Oblivious Transfers depend on the Executor's input size whilst the sub-computation depends on the Builder's input size, together these two components account for $>90\%$ of the running time on the addition circuit. If this trend holds then this protocol would be well suited to large circuits with small input sizes.\\

				Also noteworthy is that the computational costs of the protocol are fairly well balanced, with only a slight skewing towards the Builder. Similarly for the communications, however as already stated we expect the main computation circuits sending to unbalance this as the circuit sizes increase.\\

				\pagebreak
				\FloatBarrier
				\noindent \textbf{32-bit Multiplication}
				\begin{figure}[!ht]
					\begin{tabular}{| p{4.3cm} | c c c c |}
						\hline
						\textbf{Builder} & \textbf{CPU Time} & \textbf{Wall Time} & \textbf{Bytes Sent} & \textbf{Bytes Recv} \\
						\thickhline
						Input generation & $2.36$ & $0.30$ & $77$ & $77$ \\
						\hline
						Circuit Building & $3.53$ & $0.46$ & $0$ & $0$ \\
						\hline
						OT- Sender & $37.59$ & $8.94$ & $284,040$ & $135,783$ \\
						\hline
						Sending Circuits and Hash List & $0.14$ & $0.54$ & $62,157,592$ & $0$ \\
						\hline
						Make/Send commits & $10.83$ & $1.38$ & $374,062$ & $0$ \\
						\hline
						Partially Open J-set & $0.06$ & $0.01$ & $134,507$ & $685$ \\
						\hline
						Sub-computation & $95.10$ & $14.66$ & $3,294,887$ & $3,162,882$\\
						\hline
						Send B-Lists & $0.00$ & $0.00$ & $2,052$ & $0$ \\
						\hline
						Prove Input Consistency & $0.00$ & $0.00$ & $22,986$ & $0$ \\
						\thickhline
						Total & $149.61$ & $26.28$ & $66,270,204$ & $3,299,428$ \\
						\hline
					\end{tabular}
					\caption{The performance of the Builder in the L-HKE 2015 protocol evaluating the 32-bit multiplication circuit averaged over 100 trials. \label{table:L-HKE_2015_Mul_Builder}}
				\end{figure}

				\begin{figure}[!ht]
					\begin{tabular}{| p{4.3cm} | c c c c |}
						\hline
						\textbf{Executor} & \textbf{CPU Time} & \textbf{Wall Time} & \textbf{Bytes Sent} & \textbf{Bytes Recv} \\
						\thickhline
						OT - Receiver & $35.29$ & $9.69$ & $135,860$ & $284,117$ \\
						\hline
						Receiving circuits and Hashed List & $0.28$ & $0.58$ & $0$ & $62,157,592$ \\
						\hline
						Receiving commits & $0.00$ & $1.35$ & $0$ & $374,062$ \\
						\hline
						Partially open J-set & $0.00$ & $0.00$ & $685$ & $134,507$ \\
						\hline
						Evaluate Circuits & $0.26$ & $0.27$ & $0$ & $0$ \\
						\hline
						Sub-computation & $94.49$ & $14.39$ & $3,162,882$ & $3,294,887$ \\
						\hline
						Verify B-List & $0.00$ & $0.00$ & $0$ & $2,052$ \\
						\hline
						Checking correctness & $3.79$ & $0.52$ & $0$ & $0$ \\
						\hline
						Verify input consistency & $0.93$ & $0.12$ & $0$ & $22,986$ \\
						\thickhline
						Total & $135.05$ & $26.92$ & $3,299,428$ & $66,270,204$ \\
						\hline
					\end{tabular}
					\caption{The performance of the Executor in the L-HKE 2015 protocol evaluating the 32-bit multiplication circuit averaged over 100 trials. \label{table:L-HKE_2015_Mul_Executor}}
				\end{figure}
				\FloatBarrier

				These results give further evidence to support the claim that size of the circuit is a small factor in the running time compared to the input sizes. Despite an increase in the circuit size by a factor of $30$ the running time increase by about $5\%$.\\

				The communications costs are where the most change occurs. As we thought the communications costs of sending the circuits has risen to dominate that of the sub-computation.\\

				\FloatBarrier
				\noindent \textbf{AES Encryption}
				\begin{figure}[!ht]
					\begin{tabular}{| p{4.3cm} | c c c c |}
						\hline
						\textbf{Builder} & \textbf{CPU Time} & \textbf{Wall Time} & \textbf{Bytes Sent} & \textbf{Bytes Recv} \\
						\thickhline
						Input generation & $9.39$ & $1.18$ & $77$ & $77$ \\
						\hline
						Building circuits & $11.18$ & $1.44$ & $0$ & $0$ \\
						\hline
						OT- Sender & $141.29$ & $33.36$ & $1,120,488$ & $476,293$ \\
						\hline
						Sending circuits and Hash List & $0.40$ & $1.82$ & $204,082,568$ & $0$ \\
						\hline
						Make/Send commits & $43.23$ & $5.44$ & $1,495,342$ & $0$ \\
						\hline
						Partially open J-set & $0.22$ & $0.03$ & $414,781$ & $684$ \\
						\hline
						Sub-computation & $212.13$ & $34.95$ & $7,516,036$ & $7,391,121$ \\
						\hline
						Send B-Lists & $0.00$ & $0.00$ & $4,100$ & $0$ \\
						\hline
						Prove input consistency & $0.00$ & $0.00$ & $92,025$ & $0$ \\
						\thickhline
						Total & $417.84$ & $78.22$ & $214,725,419$ & $7,868,176$ \\
						\hline
					\end{tabular}
					\caption{The performance of the Builder in the L-HKE 2015 protocol evaluating the AES encryption circuit averaged over 100 trials. \label{table:L-HKE_2015_AES_Builder}}
				\end{figure}

				\begin{figure}[!ht]
					\begin{tabular}{| p{4.3cm} | c c c c |}
						\hline
						\textbf{Executor} & \textbf{CPU Time} & \textbf{Wall Time} & \textbf{Bytes Sent} & \textbf{Bytes Recv} \\
						\thickhline
						OT - Receiver & $136.87$ & $35.96$ & $476,370$ & $1,120,565$ \\
						\hline
						Receiving circuits and Hashed List & $0.76$ & $1.88$ & $0$ & $204,082,568$ \\
						\hline
						Receiving commits & $0.02$ & $5.43$ & $0$ & $1,495,342$ \\
						\hline
						Partially open J-set & $0.01$ & $0.01$ & $684$ & $414,781$ \\
						\hline
						Evaluate circuits & $0.37$ & $0.37$ & $0$ & $0$ \\
						\hline
						Sub-computation & $211.41$ & $34.57$ & $7,391,121$ & $7,516,036$ \\
						\hline
						Verify B-List & $0.00$ & $0.00$ & $0$ & $4,100$ \\
						\hline
						Checking correctness & $11.14$ & $1.80$ & $0$ & $0$ \\
						\hline
						Verify input consistency & $2.89$ & $0.47$ & $0$ & $92,025$ \\
						\thickhline
						Total & $363.46$ & $80.49$ & $7,868,176$ & $214,725,419$ \\
						\hline
					\end{tabular}
					\caption{The performance of the Executor in the L-HKE 2015 protocol evaluating the AES encryption circuit averaged over 100 trials. \label{table:L-HKE_2015_AES_Executor}}
				\end{figure}





\end{document}
